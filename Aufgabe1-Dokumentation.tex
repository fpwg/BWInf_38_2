\documentclass[a4paper,10pt,ngerman]{scrartcl}
%\usepackage{babel}
%\usepackage[utf8]{inputenc}
\usepackage[T1]{fontenc}
\usepackage[a4paper,margin=2.5cm,footskip=0.5cm]{geometry}

% Die nöchsten drei Felder bitte anpassen:
\newcommand{\Aufgabe}{Aufgabe 1: Stromrallye} % Aufgabennummer und Aufgabennamen angeben
\newcommand{\TeilnahmeId}{55252}       % Teilnahme-Id angeben
\newcommand{\Namen}{Florian Plaswig} % Namen der Bearbeiter/-innen dieser Aufgabe angeben


% Kopf- und Fuözeilen
\usepackage{scrlayer-scrpage, lastpage}
\setkomafont{pageheadfoot}{\large\textrm}
\lohead{\Aufgabe}
\rohead{Teilnahme-Id: \TeilnahmeId}
\cfoot*{\thepage{}/\pageref{LastPage}}

% Position des Titels
\usepackage{titling}
\setlength{\droptitle}{-1.0cm}

% För mathematische Befehle und Symbole
\usepackage{amsmath}
\usepackage{amssymb}

% För Bilder
\usepackage{graphicx}

% För Algorithmen
\usepackage{algpseudocode}

\let\oldReturn\Return
\renewcommand{\Return}{\State\oldReturn}

% För Quelltext
\usepackage{listings}
\usepackage{color}
\definecolor{mygreen}{rgb}{0,0.6,0}
\definecolor{mygray}{rgb}{0.5,0.5,0.5}
\definecolor{mymauve}{rgb}{0.58,0,0.82}
\lstset{
  keywordstyle=\color{blue},commentstyle=\color{mygreen},
  stringstyle=\color{mymauve},rulecolor=\color{black},
  basicstyle=\footnotesize\ttfamily,numberstyle=\tiny\color{mygray},
  captionpos=b, % sets the caption-position to bottom
  keepspaces=true, % keeps spaces in text
  numbers=left, numbersep=5pt, showspaces=false,showstringspaces=true,
  showtabs=false, stepnumber=2, tabsize=2, title=\lstname
}
\lstdefinelanguage{JavaScript}{ % JavaScript ist als einzige Sprache noch nicht vordefiniert
  keywords={break, case, catch, continue, debugger, default, delete, do, else, finally, for, function, if, in, instanceof, new, return, switch, this, throw, try, typeof, var, void, while, with},
  morecomment=[l]{//},
  morecomment=[s]{/*}{*/},
  morestring=[b]',
  morestring=[b]",
  sensitive=true
}

% Diese beiden Pakete mössen zuletzt geladen werden
%\usepackage{hyperref} % Anklickbare Links im Dokument
\usepackage{cleveref}

% Daten för die Titelseite
\title{\textbf{\Huge\Aufgabe}}
\author{\LARGE Teilnahme-Id: \LARGE \TeilnahmeId \\\\
	    \LARGE Bearbeiter/-in dieser Aufgabe: \\
	    \LARGE \Namen\\\\}
\date{\LARGE\today}

\begin{document}

\maketitle
\tableofcontents

\vspace{0.5cm}

\section{Lösungsidee}

\subsection{Interne Representation}
Das Spielfeld wird intern als Graph repräsentiert. Dabei ist jede Position auf dem \(n\times n\)-Feld einem Knoten zugeordnet, welcher mit seinen 4 Nachbarn verbunden ist. Da sowieso Bewegungsfreiheit herrscht, ist diese Form der Darstellung in der Standardvariante des Spiels nicht notwendig, bei Erweiterungen des Spiels vereinfacht sie jedoch deren Implementation. Beispielsweise können neue Elemente wie Portale, Einwegstrecken und Mauern ausschlieölich durch Änderungen des Graphen implementiert werden.

Der Graph wird durch ein assoziatives Array (Dictionary) repräsentiert. Darin sind die einzelnen Knoten/Positionen die Schlüssel, denen eine Liste der verbundenen Knoten zugeordnet ist. Die Positionen der Kacheln werden als Integer-Tupel gespeichert. Es wöre ebenso möglich die Kacheln zu nummerieren, anstatt ihre Positionen zu verwenden doch dies hätte in der konkreten Implementation keinen Vorteil und würde die grafische Ausgabe verkomplizieren.

Den Positionen der Batterien werden in einem assoziativen Array ihre Ladungen zugeordnet.

\subsection{Lösbarkeit und Lösungsbestimmung} \label{solution}
Das Grundprinzip, mit dem die Lösbarkeit bestimmt wurde basiert auf Backtracking. In einer rekursiven Funktion wird der Graph \(G(P)\) schrittweise durchwandert und der gegangene Weg \(W\), sowie die Position \(P_R\), verönderte Batteriewerte \(B_n\) und Bordbatteriewerte \(B_R\) werden in jedem rekursiven Aufruf übergeben. Sollte die Bordbatterie aber nicht die anderen Batterien leer sein, so wird der bisher gegangene Weg aufgegeben.
Wenn ein Zustand erreicht wird, in dem alle Batterien leer sind, so konnte eine mögliche Lösung gefunden werden. Diese wird zurückgeben und das Programm endet. Geschieht dies nicht, so wird nichts zurückgegeben, es wurde keine Lösung gefunden. Eine Lösung wird dargestellt als Liste von Integer-Tupeln, welche die Knoten/Kacheln beschreiben, welche abgelaufen werden mössen, um das Puzzle zu lösen.
\\
\begin{algorithmic}[1]
\Procedure{LöseStromrallye}{$G, W, P_{curr}, B_R,  B$}
	\If{$B_R = 0$ und $\sum B_n = 0$}\Comment Lösung gefunden
		\Return $P_R$
	\ElsIf {$B_R = 0$ und $\sum B_n > 0$} \Comment Sackgasse
		\Return $\varnothing$
	\EndIf
	\ForAll{$P_{next} \in G(P_{curr})$}
		\If{$P_{next}$ auf Batteriefeld}
			\State Tausche Batterien\footnote{Die Bordbatterie wird nicht mit leeren Batterien getauscht}
		\EndIf
		\State $Resultat \gets \textsc{LöseStromrallye}(G, W + P_{curr}, P_{next}, B_R,  B)$
		\If{$Resultat \neq \varnothing$} \Comment Gelöst
			\Return $Resultat$
		\EndIf
	\EndFor
	\Return $\varnothing$ \Comment Es gibt keine Lösung
\EndProcedure
\end{algorithmic}

\subsection{Definition der Schwierigkeit}
Schwierigkeit steigt für einen menschlichen Spieler mit der Anzahl der zur Verfügung stehenden Handlungsoptionen bzw. Züge die durchdacht werden müssen und den Spieler in die Irre führen. Sie sinkt mit der Anzahl der Zöge die zum richtigen Ergebnis führen - je mehr Arten es gibt, auf die das Rätsel gelöst werden kann, desto einfacher wird es also.
Als Maß der Schwierigkeit $d$ kann man den Quotienten aus in eine Sackgasse $n_{falsch}$ und zur Lösung föhrenden Spielabläufen $n_{richtig}$  verwenden.
\begin{equation}
d = \frac{n_{falsch}}{n_{richtig}}
\end{equation}
Dieses Modell ist jedoch nicht vollständig korrekt sondern bietet nur eine Näherung. Ein realer Spieler kann oft die meisten Irrwege intuitiv ausschließen. Diese Wahlmöglichkeiten beeinflussen die Schwierigkeit nicht maßgeblich, da sie direkt verworfen werden. Wie gut diese Erkennung seitens des Spielers funktioniert und worauf geachtet wird hängt demnach stark von der Denkweise und Übung dessen ab. \\
Trotz alledem bietet das Modell eine relativ zuverlässige Vorhersage über die Komplexität des Puzzles, wie sich später auch an den mithilfe dieses Maßes erzeugten und beurteilten Spielsituationen erkennen lässt.

\subsection{Ermittlung der Schwierigkeit}
Der Algorithmus zur Ermittlung der Schwierigkeit durchlöuft den Graphen vollstöndig. Er ähnelt dem in \ref{solution} beschrieben Verfahren, hat jedoch andere Abbruchbedingungen und Rückgabewerte. \\
Gelangt der Algorithmus an einen Punkt, von dem aus keine weiteren Schritte ausgeführt werden können, so wird abhüngig ob der Zustand eine Lösung darstellt oder nicht $n_{richtig}$ oder respektive $n_{falsch}$ inkrementiert. Durchlöuft der Algorithmus nun so alle möglichen Zugfolgen ist es möglich mit $n_{richtig}$ und $n_{falsch}$ die Schwierigkeit einer Spielsituation zu bewerten.
\\
\begin{algorithmic}[1]
\Procedure{BewerteSchwierigkeit}{$G, P_{curr}, B_R,  B$}
	\If{$B_R = 0$ und $\sum B_n = 0$}\Comment Lösung gefunden
		\State  erhöhe $n_{richtig}$
	\ElsIf {$B_R = 0$ und $\sum B_n > 0$} \Comment Sackgasse
		\State  erhöhe $n_{falsch}$
	\EndIf
	\ForAll{$P_{next} \in G(P_{curr})$}
		\If{$P_{next}$ auf Batteriefeld}
			\State Tausche Batterien
		\EndIf
		\State $\textsc{BewerteSchwierigkeit}(G, P_{next}, B_R,  B)$
	\EndFor
\EndProcedure
\end{algorithmic}

\subsection{Generierung - 1. Ansatz Gradientenabstieg}
Bei dem Problem, ein möglichst schweres Level zu erstellen handelt es sich um ein klassisches Optimierungsproblem, bei dem das Minimum einer Funktion $\textsc{BewerteSchwierigkeit}(G, P_{curr}, B_R,  B)$ durch Anpassung der Eingabe gefunden oder angenöhert werden muss. Daher fiel die Wahl zunächst auf das Gradientenabstiegsverfahren.
Dieses Verfahren veröndert den Eingabevektor einer Funktion $f(\vec{x})$ entlang/entgegen des Gradienten $\nabla f(\vec{x})$, um ein lokales Maximum/Minimum $f(\vec{x}_{min})$ zu finden. Unglöcklicherweise ergeben sich dabei einige Probleme. \\
Es gibt viele Spielzustönde, welche nicht lösbar sind und an denen sich durch die nicht definierte ($0^{-1}$) bzw. unendliche Schwierigkeit kein Gradient bilden lösst. So eignen sich nur wenige Startzustönde, die auch erst ermittelt werden mössen.
Ebenfalls ist problematisch, dass es nur ganzzahlige Parameter gibt und nur wenige bis keine Verönderungen der Schwierigkeit durch eine Verönderung des Zustandes auftreten. Daher ist das Gradientenabstiegsverfahren zur Optimierung ungeeignet.

\subsection{Generierung - 2. Ansatz Backtracking}
Der Ausgangspunkt des zweiten Ansatzes ist eine gegebene oder zufällig bestimmte Startposition $P_{Start}$ und Startladung $B_R$ sowie 

\section{Umsetzung}
Hier wird kurz erläutert, wie die Lösungsidee im Programm tatsächlich umgesetzt wurde. Hier können auch Implementierungsdetails erwähnt werden.

\section{Beispiele}
Genügend Beispiele einbinden! Die Beispiele von der BwInf-Webseite sollten hier diskutiert werden, aber auch eigene Beispiele sind sehr gut, besonders wenn sie Spezialfälle abdecken. Aber bitte nicht 30 Seiten Programmausgabe hier einfügen!

\section{Quellcode}
Unwichtige Teile des Programms sollen hier nicht abgedruckt werden. Dieser Teil sollte nicht mehr als 2-3 Seiten umfassen, maximal 10.


\end{document}
